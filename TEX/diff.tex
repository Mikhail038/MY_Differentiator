
    \documentclass[a4paper, 12pt]{article}
    \usepackage{geometry}
    \geometry{a4paper,
    total={170mm,257mm},left=2cm,right=2cm,
    top=2cm,bottom=2cm}
    \usepackage{setspace}
    \usepackage{color}
    \usepackage{hyperref}
    \usepackage{mathtext}
    \usepackage{amsmath}
    \usepackage[utf8]{inputenc}
    \usepackage[english,russian]{babel}
    \usepackage{graphicx, float}
    \usepackage{tabularx, colortbl}
    \usepackage{textcomp}
    \usepackage{caption}
    \usepackage{wrapfig}
    \usepackage{multirow}
    \usepackage{subfigure}

    \DeclareMathOperator{\sgn}{\mathop{sgn}}
    \newcommand*{\hm}[1]{#1\nobreak\discretionary{}
        {\hbox{$\mathsurround=0pt #1$}}{}}


    \author{Рачков Михаил Васильевич, Б01-201}
    \date{Nov 30 2022}
    \title{\textbf{Отчёт о взятии производной} \\(После взятия производная была положена обратно)}


    \captionsetup{labelsep=period}

    \newcommand{\parag}[1]{\paragraph*{#1:}}
    \newcounter{Points}
    \setcounter{Points}{1}
    \newcommand{\point}{\noindent \arabic{Points}. \addtocounter{Points}{1}}
    \newcolumntype{C}{>{\centering\arraybackslash}X}


    \begin{document}

        \maketitle 
            \maketitle
            \begin{equation}
            f(x)~=~x ^ {2} + x ^ {2}
    \end{equation}
    \\ Упростим по возможности: \\
            \maketitle
            \begin{equation}
            f(x)~=~x ^ {2} + x ^ {2}
    \end{equation}
    \\ Возьмём производную: \\
            \maketitle
            \begin{equation}
            f'(x)~=~2 \cdot x ^ {2 - 1} \cdot 1 + 2 \cdot x ^ {2 - 1} \cdot 1
    \end{equation}
    \\ Упростим по возможности: \\
            \maketitle
            \begin{equation}
            f'(x)~=~2 \cdot x + 2 \cdot x
    \end{equation}
    
    \section{Источники}
    Я самоучка.
    
    \end{document}
    