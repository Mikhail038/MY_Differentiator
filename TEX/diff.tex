\documentclass[a4paper, 12pt]{article}
    \usepackage{geometry}
    \geometry{a4paper,
    total={170mm,257mm},left=2cm,right=2cm,
    top=2cm,bottom=2cm}
    \usepackage{setspace}
    \usepackage{color}
    \usepackage{hyperref}
    \usepackage{mathtext}
    \usepackage{amsmath}
    \usepackage[utf8]{inputenc}
    \usepackage[english,russian]{babel}
    \usepackage{graphicx, float}
    \usepackage{tabularx, colortbl}
    \usepackage{textcomp}
    \usepackage{caption}
    \usepackage{wrapfig}
    \usepackage{multirow}
    \usepackage{subfigure}

    \DeclareMathOperator{\sgn}{\mathop{sgn}}
    \newcommand*{\hm}[1]{#1\nobreak\discretionary{}
        {\hbox{$\mathsurround=0pt #1$}}{}}


    \author{Рачков Михаил Васильевич, Б01-201}
    \date{Dec  7 2022}
    \title{\textbf{Отчёт о взятии производной} \\(После взятия производная была положена обратно)}


    \captionsetup{labelsep=period}

    \newcommand{\parag}[1]{\paragraph*{#1:}}
    \newcounter{Points}
    \setcounter{Points}{1}
    \newcommand{\point}{\noindent \arabic{Points}. \addtocounter{Points}{1}}
    \newcolumntype{C}{>{\centering\arraybackslash}X}


    \begin{document}

        \maketitle 
            \maketitle
            \begin{equation}
            f(x)~=~x ^ {x} \cdot \dfrac{ \sin x -  \cosh x ^ {A} }{ B}
    \end{equation}
    
    Где: \\
    
            \begin{equation}
            A =~\dfrac{ \tan x }{  \sinh x ^ {2}}
            \end{equation}
            
            \begin{equation}
            B =~ \sin (x ^ {4}) + x ^ {5}
            \end{equation}
            \\ Упростим по возможности: \\
            \maketitle
            \begin{equation}
            f(x)~=~x ^ {x} \cdot \dfrac{ \sin x -  \cosh x ^ {A} }{ B}
    \end{equation}
    
    Где: \\
    
            \begin{equation}
            A =~\dfrac{ \tan x }{  \sinh x ^ {2}}
            \end{equation}
            
            \begin{equation}
            B =~ \sin (x ^ {4}) + x ^ {5}
            \end{equation}
            \\ Возьмём производную: \\
            \maketitle
            \begin{equation}
            f'(x)~=~x ^ {x} \cdot A \cdot \dfrac{ \sin x -  \cosh x ^ {B} }{ C} + \dfrac{D }{ N ^ {2}} \cdot x ^ {x}
    \end{equation}
    
    Где: \\
    
            \begin{equation}
            A =~(1 \cdot  \ln x + \dfrac{1 }{ x} \cdot x)
            \end{equation}
            
            \begin{equation}
            B =~\dfrac{ \tan x }{  \sinh x ^ {2}}
            \end{equation}
            
            \begin{equation}
            C =~ \sin (x ^ {4}) + x ^ {5}
            \end{equation}
            
            \begin{equation}
            D =~E \cdot F \cdot H \cdot I - (J + L \cdot 1) \cdot  \sin x -  \cosh x ^ {M}
            \end{equation}
            
            \begin{equation}
            E =~\dfrac{ \tan x }{  \sinh x ^ {2}}
            \end{equation}
            
            \begin{equation}
            F =~ \sin x -  \cosh x ^ {G - 1}
            \end{equation}
            
            \begin{equation}
            G =~\dfrac{ \tan x }{  \sinh x ^ {2}}
            \end{equation}
            
            \begin{equation}
            H =~( \cos x \cdot 1 -  \sinh x \cdot 1)
            \end{equation}
            
            \begin{equation}
            I =~( \sin (x ^ {4}) + x ^ {5})
            \end{equation}
            
            \begin{equation}
            J =~ \cos (x ^ {4}) \cdot K \cdot 1
            \end{equation}
            
            \begin{equation}
            K =~4 \cdot x ^ {4 - 1}
            \end{equation}
            
            \begin{equation}
            L =~5 \cdot x ^ {5 - 1}
            \end{equation}
            
            \begin{equation}
            M =~\dfrac{ \tan x }{  \sinh x ^ {2}}
            \end{equation}
            
            \begin{equation}
            N =~ \sin (x ^ {4}) + x ^ {5}
            \end{equation}
            \\ Упростим по возможности: \\
            \maketitle
            \begin{equation}
            f'(x)~=~x ^ {x} \cdot A \cdot \dfrac{ \sin x -  \cosh x ^ {B} }{ C} + \dfrac{D }{ L ^ {2}} \cdot x ^ {x}
    \end{equation}
    
    Где: \\
    
            \begin{equation}
            A =~( \ln x + \dfrac{1 }{ x} \cdot x)
            \end{equation}
            
            \begin{equation}
            B =~\dfrac{ \tan x }{  \sinh x ^ {2}}
            \end{equation}
            
            \begin{equation}
            C =~ \sin (x ^ {4}) + x ^ {5}
            \end{equation}
            
            \begin{equation}
            D =~E \cdot I - (J + 5 \cdot x ^ {4}) \cdot  \sin x -  \cosh x ^ {K}
            \end{equation}
            
            \begin{equation}
            E =~F \cdot G \cdot ( \cos x -  \sinh x)
            \end{equation}
            
            \begin{equation}
            F =~\dfrac{ \tan x }{  \sinh x ^ {2}}
            \end{equation}
            
            \begin{equation}
            G =~ \sin x -  \cosh x ^ {H - 1}
            \end{equation}
            
            \begin{equation}
            H =~\dfrac{ \tan x }{  \sinh x ^ {2}}
            \end{equation}
            
            \begin{equation}
            I =~( \sin (x ^ {4}) + x ^ {5})
            \end{equation}
            
            \begin{equation}
            J =~ \cos (x ^ {4}) \cdot 4 \cdot x ^ {3}
            \end{equation}
            
            \begin{equation}
            K =~\dfrac{ \tan x }{  \sinh x ^ {2}}
            \end{equation}
            
            \begin{equation}
            L =~ \sin (x ^ {4}) + x ^ {5}
            \end{equation}
            
    \section{Источники}
    Я самоучка.
    
    \end{document}
    